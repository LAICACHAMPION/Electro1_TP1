

\documentclass[../../e1_tp1_main.tex]{subfiles}

\begin{document}

\newgeometry{left =1.5cm, right=1.5cm, top=1.5cm, bottom=2cm}

\chapter{}


En este ejercicio, se estudi\'o el comportamiento de un transistor NPN en el siguiente circuito con emisor com\'un:

\begin{figure}[H]

	\centering
 	\begin{circuitikz}
 	\draw 
 	(3.5,1) node[npn] (npn) {}
 
 	(npn.B) to [short, -*] (2.0,1)
 	to [C, l_=$C_{in}$, -*] (0.5,1) node[above]{$V_{in}$}
 	to [R, l_=$R_G$] (-1.5,1)
 	to [american voltage source=$V_{G}$] (-1.5,-1.5) 
 	
 	(2,1) to [R, l_=$R_2$] (2.0,-1.5)
 	(2,1) to [R=$R_1$] (2.0, 3.5)
 	to [short] (-1.5,3.5)
 	to[american voltage source = $V_{CC}$] (-1.5,2) node[ground]{}
 	
 	(npn.E) to [R, l_= $R_E$] (3.5,-1.5)
 	(3.5,0.5) to [short, *-](5,0.5)
 	to [C, l_=$C_E$](5,-1.5)
 	
 	(npn.C) to [R, l_=$R_C$] (3.5,3.5)
 	to [short, -*] (2,3.5)
 	
 	(3.5,1.5) to [C=$C_{out}$, *-*] (6,1.5) node[right]{$V_{out}$}
 	to [R= $R_L$] (6,-1.5)
 	to [short, -*] (5,-1.5)
 	to [short, -*] (3.5,-1.5)
 	to [short] (-1.5,-1.5) node[ground]{}
 	
 	;\end{circuitikz}
 	
 	\caption{Esquema del circuito}
\end{figure}

En particular, el modelo de transistor utilizado fue el BC547\footnote{La hoja de datos de este transistor puede encontrarse en el siguiente link: \url{ https://www.sparkfun.com/datasheets/Components/BC546.pdf}}.\par

Los valores de los componentes pasivos son los siguientes:


\begin{table}[H]
	\centering
	\begin{tabular}{|c|c|c|c|c|c|c|c|}
	\hline
	$R_1$        &	$R_2$	& $R_C$        & $R_E$       & $R_L$       & $C_{in}$ & $C_{out}$ & $C_{E}$  \\ \hline
	$100k\Omega$ & $8.2k\Omega$ & $5.6k\Omega$ & $250\Omega$ & 10$k\Omega$ & 10$nF$   & $10nF$    & $1\mu F$ \\ \hline
	\end{tabular}
	\caption{Valores de las resistencias y los capacitores utilizados}
\end{table}


La resistencia $R_G$ se encuentra excluida de estas consideraciones puesto a que es la resistencia interna del generador, cuyo valor est\'andar es $50\Omega$. A su vez, el transistor se aliment\'o con $V_{CC}=12V$ \par



\section{An\'alisis te\'orico}

\subsection{Polarizaci\'on}
Como la polarizaci\'on del transistor se realiza en continua, no circula corriente por las ramas del circuito donde hay capacitores. Por lo tanto, se puede simplificar de la siguiente manera:

\begin{figure}[H]
	\centering
 	\begin{circuitikz}
 	\draw 
 	(3.5,1) node[npn] (npn) {}

 	(npn.B) to [short, -*] (2.0,1) 	
 	to [R, l_=$R_2$] (2.0,-1.5) node[ground]{}
 	
 	(2,1) to [R=$R_1$] (2.0, 3.5)
 	to [short] (-1.5,3.5)
 	to[american voltage source = $V_{CC}$] (-1.5,2) node[ground]{}
 	
 	(npn.E) to [R= $R_E$] (3.5,-1.5)
 	to [short, -*](2.0,-1.5)
 	
 	(npn.C) to [R, l_=$R_C$] (3.5,3.5)
 	to [short, -*] (2,3.5)
 	

 	;\end{circuitikz}
 	
 	\caption{Esquema del circuito en continua}
\end{figure}

Aplicaremos el teorema de Th\'evenin para resolverlo, es decir para obtener $I_{B_q} = $. Aplicando un divisor resistivo para la tensi\'on de Th\'evenin, y pasivando $V_{CC}$ y considerando que por la resistencia $R_E$ pasa $I_E=I_C+I_B=I_B\cdot(h_{FE}+1)$, el circuito resultante es el siguiente:\par

\begin{figure}[H]
\centering
	\label{fig:th}
 	\begin{subfigure}[t]{0.5\linewidth}
 		\begin{circuitikz}
 		\draw 
	 	(3.5,1) node[npn] (npn) {}
 		(npn.B) to [R=$R_{Th}$] (-1,1)
 		to [american voltage source = $V_{Th}$] (-1,-1.5) node[ground]{}
 
 		(npn.E) to [R, l_= $R_E \cdot (1+h_{FE})$] (3.5,-1.5)
 		to [short, -*](-1,-1.5)
 		;\end{circuitikz}
 	\end{subfigure}

	\begin{subfigure}[t]{0.2\linewidth}

	\end{subfigure}

 	\caption{Circuito de Th\'evenin en continua}
\end{figure}

En la figura (\ref{fig:th}), $R_{Th} =\frac{R_1 \cdot R_2}{R_1+R_2} = ...$ y $V_{Th} = \left( \frac{R_2}{R_1 + R_2} \right) \cdot V_{CC} = ...$. De esta manera obtenemos que los valores de las corrientes de polarizaci\'on son:

	
 \[\left\{
\begin{aligned}
		I_{Bq} &=\frac{V_{Th}-V_{BE_{ON}}}{R_{Th} + R_E \cdot (1+h_{FE})} &= ... \\
		\\
		I_{Cq} &= h_{FE} \cdot I_{Bq} =  \frac{V_{Th}-V_{BE_{ON}}}{ \frac{R_{Th}}{h_{FE}} + R_E \cdot 						\left(\frac{1+h_{FE}}{h_{FE}}\right)} &= ...
 \end{aligned}
 \right.\]
 	
\subsection{Modelo incremental}

\subsection{Circuito incremental}

Aca se deber\'ia concluir el comportamiento en 


\end{document}

