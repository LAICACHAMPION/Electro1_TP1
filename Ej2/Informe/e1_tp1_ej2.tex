

\documentclass[../../e1_tp1_main.tex]{subfiles}

\begin{document}

\newgeometry{left =1.5cm, right=1.5cm, top=1.5cm, bottom=2cm}

\chapter{}


En este ejercicio, se estudi\'o el comportamiento de un transistor NPN en el siguiente circuito:

\begin{figure}[H]

	\centering
 	\begin{circuitikz}
 	\draw 
 	(3.5,1) node[npn] (npn) {}
 	
 %	(-1,-1.5) to [american voltage source=$V_{G}$] (-1,1)
 %	to [R, l_=$R_G$](1,1) node[above]{$V_{in}$}
 %	to [C, l_=$C_{in}$](2.5,1)
 %	to [short, *-] (npn.B) 
 	(npn.B) to [short, -*] (2.5,1)
 	to [C, l_=$C_{in}$] (1,1) node[above]{$V_{in}$}
 	to [R, l_=$R_G$] (-1,1)
 	to [american voltage source=$V_{G}$] (-1,-1.5) 
 	
 	
 	(2.5,1) to [R=$R_1$] (2.5, 3.5)
 	to [short] (-1,3.5)
 	to[american voltage source = $V_{CC}$] (-1,2) node[ground]{}
 	
 	(npn.E) to [R, l_= $R_E$] (3.5,-1.5)
 	(3.5,0.5) to [short, *-](5,0.5)
 	to [C, l_=$C_E$](5,-1.5)
 	
 	(npn.C) to [R, l_=$R_C$] (3.5,3.5)
 	to [short, -*] (2.5,3.5)
 	
 	(3.5,1.5) to [C=$C_{out}$, *-] (6,1.5)
 	to [R= $R_L$] (6,-1.5)
 	to [short, -*] (5,-1.5)
 	to [short, -*] (3.5,-1.5)
 	to [short] (-1,-1.5) node[ground]{}
 	
 	;\end{circuitikz}
 	
 	\caption{Esquema del circuito}
\end{figure}

En particular, el modelo de transistor utilizado fue el BC547\footnote{La hoja de datos de este transistor puede encontrarse en el siguiente link: \url{ https://www.sparkfun.com/datasheets/Components/BC546.pdf}}.\par

En cuanto a los dem\'as componentes, los valores fueron los siguientes:

\todo[inline]{valores pedidos por la consigna vs medidos}

La resistencia $R_G$ se encuentra excluida de estas consideraciones puesto a que es la resistencia interna del generador, cuyo valor est\'andar es $50\Omega$.\par

\end{document}
